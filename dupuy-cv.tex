\documentclass[a4paper,10pt]{article}
\usepackage{fullpage,amsmath,amssymb,hyperref}
\usepackage{natbib}
\usepackage{bibentry}
\usepackage{multicol}
\usepackage{enumitem}


\newcommand{\CC}{\mathbf{C}}
\newcommand{\ZZ}{\mathbf{Z}}
\newcommand{\alg}{\operatorname{alg}}
%opening
\begin{document}

\bibliographystyle{plain}
\nobibliography{statement}
\begin{multicols}{2}
\begin{flushleft}
    {\Large \bf Taylor Dupuy, PhD}\\
	Department of Mathematics and Statistics \\
	82 University Place, Burlington, VT, 05401  
	\newline \newline
	\begin{tabular}{ll}
		email: & tdupuy@uvm.edu\\
		web: & \url{www.uvm.edu/~tdupuy}\\
		YouTube: & \url{youtube.com/agexplained} \\
		Github: & \url{github.com/tdupu} 
	\end{tabular}
\end{flushleft}
\end{multicols}

 
 
 \subsection*{Appointments}
 \vspace*{-10pt}
\begin{center}
\begin{tabular}{p{1.7in} p{3.5in}p{1in}}
Assistant Professor & University of Vermont & 2022 -- \\
Graduate Program Director & University of Vermont & 2021 \\
Assistant Professor & University of Vermont & 2018--2021 \\
Visiting Assistant Professor & University of Vermont & 2016--2018 \\
Postdoctoral Fellow & The Hebrew University& 2014--2016 \\
Postdoctoral Fellow & Mathematical Sciences Research Institute & 2014 \\
Adjunct Assistant Professor & UCLA & 2013--2014 \\
%Teaching Assistant & University of New Mexico, Department of Mathematics & 2007--2013 \\
%Research Assistant & University of Arizona, Department of Biochemistry & 2006--2007\\
%Research Assistant & University of Arizona, Department of Mathematics & 2005--2006\\
\end{tabular}
\end{center}

 \subsection*{Education}
 \begin{flushleft}
\begin{tabular}{p{2in} p{1in}p{1in}}
University of Tulsa & Mathematics & 2003--2004 \\
University of Arizona, BS & Mathematics & 2004--2007\\
University of New Mexico, PhD &  Mathematics & 2007--2013
 \end{tabular}
 \end{flushleft}

\iffalse 
\subsection*{Research Interests}
Broad:  Algebraic Geometry, Differential Algebra, Number Theory, Applied Model Theory  \\
Specific: Deformation Theory, Diophantine Geometry, Differential Algebra and Witt vectors.
\fi 

\section*{Research Activities}

\subsection*{Publications}
\begin{enumerate}
	\item \bibentry{Dupuy2023}
	\item  \bibentry{Dupuy2022a} 
	\item  \bibentry{Dupuy2020b}
	\item \bibentry{Dupuy2019a}
	\item \bibentry{Dupuy2019}
	\item \bibentry{Dupuy2017b}
	\item \bibentry{Buium2016}
	\item \bibentry{Buium2016a}
	\item \bibentry{Buium2016b}
	\item \bibentry{Dupuy2016}
	\item \bibentry{Dupuy2014a}
	\item \bibentry{Dupuy2014}
\end{enumerate}

\subsection*{Preprints}
\begin{enumerate}[resume]
	\item \bibentry{Dupuy2023a}
	\item \bibentry{Dupuy2020a}
	\item \bibentry{Dupuy2020} 
	\item  \bibentry{Dupuy2017} 
	\item  \bibentry{Dupuy2017a}
\end{enumerate}



\subsection*{Honors and Awards}
\begin{enumerate} \addtolength{\itemsep}{-0.5\baselineskip}
\item \emph{``Witt vectors, deformations, and absolute geometry"} (DMS-1802012, \$39938), PI, 2018
\item \emph{``Arithmetic and algebraic differentiation: Witt vectors, number theory, and differential algebra''} (DMS-1502219), organizer on conference grant, Fall 2015  
 \item MSRI Postdoctoral Fellow (DMS-0932078), Spring 2014
\item New Mexico MCTP Summer Award (DMS-1148801), Summer 2013 
\item New Mexico Efroymson Summer Award, Summer 2013 
 \item New Mexico MCTP Summer Award (DMS-1148801), Summer 2012 
\end{enumerate}



\subsection*{Selected Invited Talks\footnote{ for a complete list visit \url{http://tdupu.github.io} }} 
\begin{enumerate} %\addtolength{\itemsep}{-0.5\baselineskip}
	\item \emph{In Introduction to The $p$-Adic Numbers}, Young Scholar’s Program, University of Illinois Chicago, Summer 2023
	\item \emph{The Theory of the Integers In $\mathbb{C}(t)^{alg}$ Interprets $\mathbb{Z}$}, Berkeley Model Theory Seminar, Spring 2023
	\item \emph{Some Computational Facts About Abelian Varieties For Nonspecialists}, Colloquium, TIMC, Fall 2022
	\item \emph{Angle Ranks of Abelian Varieties over Finite Fields}, Algebra Seminar, Carleton, Fall 2022
	\item \emph{Algebraic Relations Between Solutions of Order One Differential Equations on Curves}, Algebra Seminar, Emory, Summer 2022
	\item \emph{What Are the $p$-Adic Numbers?}, Young Scholar’s Program, University of Illinois Chicago, Summer 2022
	\item Panelist: \emph{“Red Card to Green Light: How to be a Responsible Referee.”}, Lunch in The Time of COVID, online, Spring 2022
	\item \emph{What is the ABC Conjecture?}, Mathematics and Statistics Colloquium, University of Vermont, Spring 2022
	\item \emph{Abelian Varieties in the LMFDB}, Geometry Seminar, University of New Mexico, Fall 2021
	\item \emph{Angle Ranks of Abelian Varieties over Finite Fields}, The 3rd Kyoto-Hefei Workshop on Arithmetic Geometry, Kyoto University, Fall 2021
	\item \emph{What is this? Have you seen this thing?}, Algebra and Number Theory Seminar, Dartmouth College, Spring 2021
	\item \emph{New and Old Results in Wittferential Algebraic Geometry}, AMS Special Session on Branching Out: Ramification Invariants in Algebra and Geometry, Spring 2021
	\item \emph{Isogeny Class of Abelian Varieties in the LMFDB}, Stanford Algebraic Geometry Seminar, Stanford, Fall 2020
	\item \emph{The Meaning of Mochizuki’s Inequality}, Geometry and Number Theory Seminar, University of Leiden, Fall 2020
	\item \emph{Sphere Packings in Hyperbolic Space}, Africa Math Seminar, Fall 2020
	%\itemUniversity of Ottawa Number Theory Seminar, College	CANCELLED DUE TO COVID
	\item \emph{Isogeny Classes of Abelian Varieties in the LMFDB}, Number Theory Seminar, Dartmouth College, Spring 2020
	\item \emph{How to work with Mochizuki's Inequality} (two parts), Algebraic Geometry and Number Theory Seminar, University of Arizona, Fall 2019
	\item \emph{Isogeny Classes of Abelian Varieties over Finite Fields}, Number Theory Seminar, Arizona State University, Fall 2019
	\item \emph{A Guide of Isogeny Classes in the LMFDB}, LMFDB as a Telescope, American Institute of Mathematics, Fall 2019
	\item \emph{Barrett Lectures	A User's Guide to Mochizuki's Inequality}, Barrett Lectures (Plenary Speaker), University of Tennesse Knoxville, Spring 2019
	\item \emph{Explicit Computations in IUT}, AGNT Seminar, Rice, Spring 2019
	\item \emph{Deligne Modules}, ICERM, Abelian Varieties over Finite Fields Workshop, Spring 2019
	\item \emph{The Wittfinitesimal Torelli Problem}, Tufts, Algebra Seminar, Fall 2018
	\item \emph{Mochizuki's Inequalities}, University of Connecticut, CTNT, Summer 2018
	\item \emph{Deligne-Illusie Classes as Arithmetic Kodaira-Spencer Classes}, Number Theory Seminar, Boston College, Fall 2017
	\item \emph{The Theory of $\CC[t]^{alg}$ interprets $\ZZ$}, Number Theory Seminar, University of Virginia, Fall 2017
	\item \emph{Indeterminacies in IUT}, Automorphic Forms Seminar, Purdue University, Spring 2017
	\item \emph{The Wittfinitesimal Torelli Problem}, Number Theory Seminar, University of Rochester, Spring 2017 
	\item \emph{Deforming Derivatives}, Differential Algebra Special Session, AMS Eastern Sectional Meeting, Spring 2017
	\item \emph{The Wittfinitesimal Torelli Problem}, QVNTS, McGill University, Spring 2017 
	\item \emph{The Theory of $\CC[t]^{alg}$ interprets $\ZZ$}, Number Theory Seminar, Harvard University, Spring 2017
	\item \emph{Arithmetic Kodaira Spencer Classes in the Sense of Buium}, DART 7, City University of New York, Fall 2016 
	\item \emph{Introduction to IUT2}, IUT Summit, RIMS Kyoto,Summer 2016.
	\item \emph{Multiradiality},  IUT Summit, RIMS Kyoto,Summer 2016. 
	\item \emph{Some Constructions Used in Mochizuki's IUT},  Number Theory Seminar, University of Copenhagen,  Spring 2016
	\item \emph{Kolchin Irreducibility}, Algebra Seminar, Emory University, Spring 2016 
	\item \emph{Effective Bounds for Manin-Mumford for Certain Bad Reduction Curves} (with E. Katz, J. Rabinoff and D. Zureick-Brown), Kolchin Seminar Workshop, Spring 2016
	\item \emph{Toward Mazur's Conjecture on Uniform Manin-Mumford} (with E. Katz, J. Rabinoff and D. Zureick-Brown), Linfoot Seminar, Bristol University, Spring 2016
	\item \emph{Examples of Lang-Bombieri-Noguchi outside of Mordell-Lang.} (with D. Litt),  Quebec Maine Number Theory Seminar, Maine University, Fall 2015
	%\item \emph{Examples of Lang-Bombieri-Noguchi outside of Mordell-Lang.} (with D. Litt),  Algebraic Geometry and Number Theory Seminar, Ben Gurion University, Summer 2015
	\item \emph{Strongly Minimal Sets in DCF0},  Logic and Set Theoretic Topology Seminar, Ben Gurion University, Summer 2015 (with J. Freitag and A. Royer)
	\item \emph{Examples of Lang-Bombieri-Noguchi Outside of Mordell-Lang}, S\'eminaire de G\'eom\'etrie Alg\'ebrique, Champs et Homotopie, Toulouse University, Spring 2015
	\item \emph{Derived Categories Meets Differential Algebra}, Model Theory and Applications Special Session, AMS-MAA Joint Mathematics Meetings 2015
	\item \emph{The Wittfinitesimal Torelli Problem} (3 parts), Model Theory Seminar, The Hebrew University, Fall 2014
	\item \emph{Kolchin Irreducibility} (with J. Freitag and L. E. Miller),
	Qu\'{e}bec-Maine Number Theory Conference, Fall 2014
	\item \emph{Jet Spaces and Diophantine Geometry},
	Kolchin Seminar, CUNY, Spring 2014
	\item \emph{Arithmetic Picard-Vessiot Theory} (with A. Buium),
	Differential Galois Theory Special Session, Spring Central Sectional Meeting of the AMS 2014
	\item \emph{Arithmetic Kolchin Irreducibility} (with J. Freitag and L. E. Miller), 
	Arithmetic and Differential Algebraic Special Session, Spring Western Sectional Meeting of the AMS 2014
	\item \emph{Arithmetic Kolchin Irreducibility} (with J. Freitag and L. E. Miller), UC--Berkeley Model Theory Seminar, Spring 2014
	% \item \emph{Automorphisms of the Affine Line over Nonreduced Rings},\\ University of New Mexico Geometry Seminar, Spring 2013
	\item \emph{The Meaning of ``Linearity" in Arithmetic Differential Equations}, UC--San Diego Number Theory Seminar, Fall 2013
	\item \emph{Arithmetic Kodaira-Spencer Classes}, S\'eminaire d'Arithm\'etique et G\'eom\'etrie Alg\'ebrique, University of Strasbourg, Fall 2013
	\item \emph{Absolute Geometry and Arithmetic Kodaira-Spencer Classes}, UCLA Number Theory Seminar, Fall 2013
	\item \emph{Linear Wittferential Equations}, (with A. Buium), Model Theory Seminar, UC--Berkeley, Spring 2013
	% \item \emph{A Torsor of Frobeniuses}, \\ Witt Vectors, Lifting and Descent Special Session, \\ AMS-MAA Joint Mathematical Meetings 2013
	%\item \emph{An Introduction to p-derivations and the Torsor of Frobeniuses},\\ Texas Tech University Algebra Seminar, Fall 2012
	%\item \emph{The Field with One Element}, \\ University of New Mexico Geometry Seminar, Spring 2012
	%\item \emph{Dyadic Harmonic Analysis and the $p$-adic Numbers},\\ DOC-COURSE on Harmonic Analysis, Metric Spaces and PDEs,\\ IMUS, University of Sevilla, Summer 2011 
	%\item \emph{Generalizing Kuratowski's Theorem to Higher Dimensions},\\ AMS General Session\\ AMS-MAA Joint Mathematical Meetings 2008
	% \item \emph{Partial Sums of Taylor Series and Some Steepest Descent Analysis}, \\ Arizona Mathematics Undergraduate Conference (now SUNMARC), Summer 2006 
\end{enumerate}

\subsection*{Other Academic Products}
\begin{enumerate}
	\item YouTube: 261K views, 14K hours viewed, 3.5K subscribers
	\begin{center}
		 \url{https://www.youtube.com/channel/UCHWnZ1NtJ4WvE5AHmNVXziw/}
	\end{center}
	\item LMFDB contributor: database of isogeny classes of abelian varieties over finite fields.\footnote{Isomorphism classes is in progress with Stefano Marseglia, Edgar Costa, and David Roe}
	 \begin{center}
	 	\url{http://lmfdb.xyz/Variety/Abelian/Fq/}
	 \end{center}
	\item \texttt{Sage} contributor. 
	\begin{center}
			\url{https://www.sagemath.org/}
	\end{center}
\end{enumerate}

\section*{Organization of Conferences/Meetings/Workshops}

\begin{enumerate} %\addtolength{\itemsep}{-0.5\baselineskip}
	\item co-organizer, ``Arithmetic Geometry'', Western Sectional Meeting of the AMS, October 2021
	\item Shepard (co-organizer), AGITTOC (Algebraic Geometry in The Time of COVID), Algebraic Geometry Lecture Series, Summer 2020  
	\item co-organizer, Abelian Varieties in the LMFDB. Funded by the University of Vermont CEMS PRSE. March 2019
	\item PI, ``Witt Vectors, Deformations, and Absolute Geometry'', Burlington, VT, July 2018 (DMS-1802012)
	\item co-organizer, ``Sage Days 87: $p$-adics and LMFDB',' Burlington, July 2017
	\item co-organizer, ``Kummer Classes and Anabelian Geometry'', (DMS-1519977 --- From Arithmetic Statistics to Zeta Elements II), Burlington, September 2016
	\item co-organizer, ``Algebraic Theory of Differential and Functional Equations,''  AMS-MAA Joint Mathematics Meetings, Atlanta, January 2016
	\item co-organizer, ``Arithmetic and Algebraic Differentiation,'' (DMS-1502219), \url{https://math.berkeley.edu/~scanlon/aad15.html}, Berkeley, May 2015 
	\item co-organizer, ``Arithmetic and Differential Algebraic Geometry", Western Sectional Meeting of the AMS, Albuquerque, Spring, January 2014
	\item co-organizer, ``Witt Vectors Lifting and Descent,'' AMS-MAA Joint Mathematics Meetings, San Diego, January 2013
	% \item co-organizer, Mathematics and Physics Student Seminar,\\ University of New Mexico, 2009--2010
	% \item co-chair, ``AMS General Session,'' \\ AMS-MAA Joint Mathematical Meetings 2008 
\end{enumerate}

\section*{University and Professional Service}
\subsection*{Seminars and Other Departmental Activities}

\begin{enumerate}
	\item Major advising (UVM): Fall 2018 --
	\item Algebra Qualifying Exam Committee (UVM): Fall 2018 -- Spring 2023
	\item Colloquium Committee Chair (UVM): Fall 2022 --
	\item Hiring Committee (UVM): Member, Spring 2020
	\item Graduate Committee (UVM): Member Fall 2017--Spring 2020; Associate Chair Fall 2020 -- Spring 2021, Graduate Program Director Fall 2021, Member Spring 2022 -- 
	\item Graduate Admissions (UVM): Spring 2017--Spring 2022
	\item unQVNTS (Algebra and Number Theory) Seminar (UVM): Organizer 2016--
	\item Algebraic Geometry Learning Seminar (UVM): Organizer 2018--2021
	\item Putnam Competition Committee (UVM): Chair Fall 2019-- Spring 2020; member Fall 2020-- Spring 2021
	\item Undergraduate Curriculum Committee (UVM): Fall 2016--Spring 2017 
	\item Special Curricular Activities (UVM): proposed and executed changes to the qualifying exams and first year sequence in 2020. 
	\item Mathematics Library Committee:  Chair Fall 2019 -- Spring 2020
\end{enumerate}

\subsection*{Editorial and Review Activities}
Journal Refereeing: \emph{Algebra and Number Theory}, \emph{International Journal of Number Theory}, \emph{Journal of Number Theory}, \emph{Math Reviews}, \emph{Journal für die reine und angewandte Mathematik (Crelle’s Journal)}, \emph{Journal of Commutative Algebra}, \emph{International Mathematics Research Notices}, \emph{Manuscripta Mathematica}, \emph{}

\section*{Advising/Mentoring}
\begin{enumerate}
	\item PhD theses: Anton Hilado 2018--, Jesse Franklin 2019--
	\item Undergraduate Research Advisee: Veronika Potter  2020 -- 2022
	\item Undergraduate Academic Advising (UVM): 2018--
	\item Graduate Academic Advising (UVM): 2018--
	\item AGITTOC (Algebraic Geometry in the Time of COVID) Shepherd: 2020 -- ; I help manage and organizer a learning seminar which at peak viewing had near 2000 concurrent viewers. The principal organizer for this is Ravi Vakil at Stanford.
	\begin{center}
			 \url{https://www.youtube.com/channel/UCy3u23mZE4TyW88yr6JLx9A}
	\end{center}
   \item  Arizona Winter School:  problem session leader (Colliot-Th\'{e}l\`ene group),  2015  

\end{enumerate}


\iffalse
\subsection*{Minor Talks}
\begin{enumerate}

 \item \emph{Deformations of Principal Homogeneous Spaces}, UNM Geometry Seminar, Albuquerque, New Mexico, Fall 2012
 \item \emph{ Arithmetic Deformation Theory}, UNM Geometry Seminar, Spring 2012
 \item \emph{ Deninger Cohomology Theories }, UNM Geometry Seminar, Spring 2012
 \item  \emph{Arithmetic Deformation Theory of Cycloelliptic Curves}, Geometry Seminar, University of New Mexico, Spring 2011
\item \emph{Why lifts of the Frobenius don't exist (for the most part) I and II}, Geometry Seminar, University of New Mexico, Fall 2011
 \item  \emph{Interactions between \u{C}ech Cohomology and Group Cohomology I and II}, Geometry Seminar, University of New Mexico, Fall 2010
 \item \emph{Coalgebras for Physicists}, Math and Physics Student Seminar, University of New Mexico, Spring 2010
 \item \emph{Derivations and p-Derivations I and II}, Geometry Seminar, University of New Mexico, Spring 2010
 \item \emph{Hilbert's 6th Problem }, Math and Physics Student Seminar, University of New Mexico, Fall 2008
 \item \emph{Parameter Estimation in Models of Ammonia Metabolism in Aedes-Aegypti Mosquitos }, Meisfeld-Wells Group Seminar, Department of Biochemistry, University of Arizona, Fall 2007
 \item  \emph{The Angel-Devil Problem }, Graduate Student Seminar, University of Arizona, Fall 2006
\end{enumerate}
\fi

\iffalse
\subsection*{Conference Attendance }
\begin{enumerate}
\item MSRI Spring 2014
\item Oberwolfach Seminar on Motivic Integration, October 2013
\item Stark's Retirement Conference, UCSD, September 2013
\item MSRI Seminar: Birational Geometry, Jet Spaces and  Characteristic p Methods (May 2013)
\item  Luminy Seminar: Geometry of the Frobenius (2013)
 \item Oberwolfach Seminar: Algebraic Groups and Patching (2012)
 \item Southwest Local Algebra Meeting (Lubbock, 2012)
 \item Texas Algebraic Geometry Symposium (College Station, 2012)
 \item Witt Vectors in Arithmetic Geometry and Topology (Albuquerque, 2012)
 \item IMUS Doc Course: Harmonic Analysis, Metric Spaces and PDEs (Sevilla, 2011)
 \item Western Algebraic Geometry Symposium (all from 2011--2013)
 \item Algebraic Dynamics (New York, 2010)
 \item Witt Vectors, Foliations and Absolute de Rham Cohomology (Nagoya, 2010)
 \item Lifts of the Frobenius (Leiden, 2009)
 \item Arizona Winter School (all from 2009--2013)
 \item 12th New Mexico Analysis Seminar (Albuquerque, 2009)
 \item AMS-MAA Joint Meetings (all from 2004--2014)
\end{enumerate}
\fi


\iffalse 
\subsection*{Teaching Experience}
\begin{itemize}\addtolength{\itemsep}{-0.5\baselineskip}
	\item Service Courses: Trigonometry, Precalculus, Calculus I, Calculus II (Large classes format), Calculus III, Business Calculus 
	\item Undergraduate: Foundations of Mathematics, Abstract Algebra, PDEs
	\item Graduate: Complex Analysis, Abstract Algebra/Commutative Algebra, Algebraic Topology
\end{itemize}
\fi

\section*{References}
%\vspace*{5pt}

%\begin{center}
%James Borger\\
%Centre for Mathematics and its Applications\\
%Mathematical Sciences Institute\\
%Australian National University, Canberra, ACT 0200, Australia\\
%james.borger@maths.anu.edu.au
%\end{center}

\begin{multicols}{2}
	\begin{flushleft}
		Alexandru Buium\\
		Department of Mathematics and Statistics\\
		University of New Mexico\\
		MSC03 2150\\
		1 University of New Mexico\\
		Albuquerque, New Mexico, 87131-0001\\
		buium@math.unm.edu
	\end{flushleft}
	\begin{flushleft}
		Kiran Kedlaya\\
		Department of Mathematics\\
		University of California, San Diego\\
		9500 Gilman Drive, \#0112\\
		La Jolla, California, 92093-0112\\
		kedlaya@ucsd.edu
	\end{flushleft}
\end{multicols}


\iffalse 
\begin{center}
	Thomas Scanlon\\
	Department of Mathematics\\
	University of California, Berkeley\\
	970 Evans Hall\\
	Berkeley, California, 94720-3840\\
	tws@berkeley.edu 
\end{center}
\fi 

\end{document}
